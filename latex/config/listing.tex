% Definicja stylu dla lstlisting dla języka YAML z zawijaniem linii
\lstdefinelanguage{yaml}{
    keywords={true,false,null,y,n,yes,no,on,off},
    keywordstyle=\color{blue},
    comment=[l]{\#},
    commentstyle=\color{green!50!black},
    stringstyle=\color{orange},
    sensitive=false,
    moredelim=[l]{:},
}

\lstdefinestyle{yamlstyle}{
    language=yaml, % Wybór języka YAML
    backgroundcolor=\color{white}, % Kolor tła
    basicstyle=\ttfamily\footnotesize, % Podstawowy styl czcionki
    breakatwhitespace=true, % Dziel wiersze w miejscach spacji
    breaklines=true, % Automatyczne zawijanie długich linii
    postbreak=\mbox{\textcolor{red}{$\hookrightarrow$}\space}, % Znak na początku nowej linii po zawinięciu
    captionpos=b, % Położenie podpisu (bottom)
    commentstyle=\color{green!50!black}, % Styl komentarzy
    keywordstyle=\color{blue}, % Styl słów kluczowych
    stringstyle=\color{orange}, % Styl stringów
    identifierstyle=\color{black}, % Styl identyfikatorów
    numberstyle=\tiny\color{gray}, % Styl numerów linii
    rulecolor=\color{black}, % Kolor ramki
    frame=single, % Ramka wokół kodu
    numbers=left, % Numerowanie linii po lewej stronie
    showstringspaces=false, % Nie pokazuj odstępów w stringach
    showspaces=false, % Nie pokazuj odstępów w kodzie
    showtabs=false, % Nie pokazuj tabulatorów jako spacji
    tabsize=2, % Rozmiar tabulatora w YAML zwykle 2
    title=\lstname % Wyświetl nazwę pliku jako tytuł
}

