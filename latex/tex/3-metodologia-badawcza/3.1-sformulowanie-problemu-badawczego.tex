\subsection{Sformułowanie problemu badawczego i pytań badawczych}

Głównym problemem badawczym, do którego rozwiązania dąży niniejsza praca, jest określenie optymalnych 
strategii autonomicznego zarządzania cyklem życia usług w środowisku Kubernetes, ze szczególnym uwzględnieniem 
mechanizmów skalowania i alokacji zasobów. W kontekście rosnącej złożoności aplikacji rozproszonych i potrzeby 
minimalizacji interwencji ludzkiej, kluczowe staje się zrozumienie, w jaki sposób istniejące rozwiązania radzą 
sobie z dynamicznie zmieniającymi się warunkami obciążenia oraz czy autorskie podejście, oparte na modelu 
MAPE-K, może zaoferować wyższą efektywność i adaptacyjność. Problem ten dotyka bezpośrednio wyzwań związanych 
z utrzymaniem wysokiej dostępności, optymalnego wykorzystania zasobów oraz obniżeniem kosztów operacyjnych w 
nowoczesnych architekturach chmurowych.

W celu rozwiązania postawionego problemu badawczego, sformułowano następujące szczegółowe pytania badawcze:
\begin{enumerate}

    \item \textbf{Charakterystyka i Ograniczenia Istniejących Mechanizmów}: W jaki sposób \gls{vpa}, \gls{hpa}
    oraz \gls{keda} reagują na różnorodne, dynamiczne wzorce obciążenia (np. narastające obciążenie, 
    obciążenie szczytowe, obciążenie zmienne cyklicznie) w kontekście efektywności alokacji zasobów 
    (CPU, pamięć) i stabilności usług (opóźnienia, błędy)? Jakie są ich fundamentalne ograniczenia w zakresie 
    proaktywnego skalowania i optymalizacji, które wynikają z ich wewnętrznych zasad działania?

    \item \textbf{Wydajność i Adaptacyjność Rozwiązania MAPE-K}: Czy autorskie rozwiązanie autonomicznej pętli 
    sterującej, zaprojektowane i zaimplementowane w oparciu o model \gls{mapek}, jest w stanie osiągnąć wyższą 
    efektywność wykorzystania zasobów oraz lepszą responsywność (np. niższe średnie i maksymalne opóźnienia) 
    aplikacji w środowisku Kubernetes w porównaniu do \gls{vpa}, \gls{hpa} i \gls{keda}, szczególnie w 
    scenariuszach wymagających zaawansowanej predykcji i koordynacji decyzji?

    \item \textbf{Wpływ Bazy Wiedzy na Efektywność MAPE-K}: W jakim stopniu komponent bazy wiedzy w 
    architekturze \gls{mapek}, integrujący historyczne dane o wydajności i predefiniowane polityki adaptacji, 
    przyczynia się do poprawy jakości decyzji podejmowanych przez pętlę sterującą oraz do długoterminowej 
    optymalizacji działania systemu w porównaniu do podejść czysto reaktywnych?

    \item \textbf{Kryteria Oceny i Metryki Sukcesu}: Jakie kluczowe metryki wydajnościowe (np. średnie zużycie
    CPU/RAM, procent niewykorzystanych zasobów, czasy odpowiedzi, przepustowość, liczba błędów) oraz wskaźniki
    operacyjne (np. liczba restartów Podów, stabilność skalowania) należy zastosować, aby obiektywnie porównać 
    skuteczność działania \gls{vpa}, {hpa}, \gls{keda} oraz autorskiego rozwiązania \gls{mapek} w 
    kontrolowanych warunkach eksperymentalnych?

\end{enumerate}
