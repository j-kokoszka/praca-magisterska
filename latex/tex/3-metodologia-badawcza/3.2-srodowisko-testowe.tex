\subsection{Środowisko testowe}

\subsubsection{Warstwa wirtualizacji i sprzętu}
W celu przeprowadzenia badań oraz weryfikacji przyjętych założeń projektowych przygotowano 
dedykowane środowisko testowe. Infrastruktura została zaprojektowana w oparciu o paradygmat 
\gls{iac}, co zapewnia powtarzalność procesu wdrażania oraz spójność konfiguracji.
Jako platformę bazową wykorzystano środowisko wirtualizacji Proxmox. Na jego potrzeby 
powołano klaster składający się z trzech maszyn wirtualnych o identycznej specyfikacji 
sprzętowej. Każdy z węzłów dysponuje następującymi zasobami:
\begin{itemize}
    \item Procesor (vCPU): 4 rdzenie, \item Pamięć operacyjna (RAM): 8 GB,
    \item Przestrzeń dyskowa: 100 GB.
\end{itemize}

Proces powoływania maszyn wirtualnych oraz konfiguracja warstwy sprzętowej zostały 
zautomatyzowane przy użyciu narzędzia Terraform.

\subsubsection{System operacyjny i orkiestracja}
Na wszystkich węzłach zainstalowano system operacyjny Ubuntu 24.04 LTS. Stanowi on bazę 
dla klastra Kubernetes, który został wdrożony przy użyciu lekkiej dystrybucji k3s w wersji 
\texttt{v1.33.2+k3s1}.

Instalacja oraz wstępna konfiguracja klastra zostały zrealizowane za pomocą narzędzia do 
zarządzania konfiguracją Ansible, z wykorzystaniem roli \texttt{k3s-ansible} dostępnej w 
repozytorium \footnote{\url{https://github.com/k3s-io/k3s-ansible}}. W ramach warstwy 
sieciowej klastra \gls{cni} zastosowano domyślne rozwiązanie dla k3s, czyli Flannel, co 
zapewnia stabilną komunikację między podami wewnątrz klastra.

Zarządzanie cyklem życia aplikacji oraz konfiguracją klastra odbywa się zgodnie z podejściem 
GitOps. Kluczowym elementem tej architektury jest narzędzie FluxCD, które stale monitoruje 
repozytorium kodu i automatycznie synchronizuje stan klastra z definicjami zawartymi w 
systemie kontroli wersji.

Proces dostarczania oprogramowania wspierany jest przez przygotowane potoki \gls{cicd},
które odpowiadają za budowanie, testowanie oraz przygotowanie manifestów wdrożeniowych. 
Takie podejście pozwoliło na pełną automatyzację procesu wdrażania kolejnych zasobów i 
usług w środowisku badawczym.

\begin{table}[h]
    \centering
    \caption{Specyfikacja techniczna środowiska eksperymentalnego}
    \label{tab:srodowisko_testowe}
    \vspace{0.3cm}
    \begin{tabular}{@{}ll@{}}
        %\toprule
        \textbf{Kategoria} & \textbf{Komponent / Specyfikacja} \\
        %\midrule
        Platforma wirtualizacji & Proxmox \\
        Automatyzacja infrastruktury & Terraform \\
        Konfiguracja systemu & Ansible (\texttt{k3s-ansible}) \\
        %\midrule
        Liczba węzłów & 3 maszyny wirtualne \\
        Zasoby węzła & 4 vCPU, 8 GB RAM, 100 GB HDD \\
        System operacyjny & Ubuntu 24.04 LTS \\
        %\midrule
        Orkiestrator & k3s (\texttt{v1.33.2+k3s1}) \\
        Sieć (CNI) & Flannel \\
        Model wdrażania & GitOps (FluxCD) \\
        %\midrule
        Narzędzie obciążające & k6 (uruchamiane zewnętrznie) \\
        Metodyka testów & Generowanie ruchu HTTP z konsoli lokalnej \\
        %\bottomrule
    \end{tabular}
\end{table}

\subsubsection{Instrumentarium badawcze i narzędzia pomiarowe}
W celu realizacji testów wydajnościowych oraz weryfikacji stabilności klastra pod obciążeniem,
zastosowano dedykowane narzędzia do generowania ruchu sieciowego.

Jako generator obciążenia wykorzystano narzędzie \textbf{k6}, które służy do przeprowadzania 
testów wydajnościowych. Skrypty testowe k6, przygotowane w języku JavaScript, są uruchamiane 
lokalnie z konsoli na stanowisku roboczym, a generowane przez nie żądania HTTP kierowane są 
na zewnątrz klastra Kubernetes, co realistycznie symuluje ruch pochodzący od zewnętrznych 
użytkowników. Taka metodyka pozwala na mierzenie kluczowych metryk z perspektywy klienta, w 
tym opóźnień (\textit{latency}), przepustowości (\textit{throughput}) oraz wskaźnika błędów.

Dane obciążeniowe kierowane są na docelową aplikację testową, która została zaprojektowana 
specjalnie na potrzeby pracy magisterskiej. Architektura i specyfikacja tej aplikacji zostały 
szczegółowo opisane w kolejnej sekcji dotyczącej eksperymentów.

