\newpage

% % % % % % % % % % % % % % % % % % % % % % % % % % % % % % % % % % % % % % % % % % % % % % % % % % % % % % % %

\section{Metodologia Badawcza}
Niniejszy rozdział poświęcony jest szczegółowemu opisowi metodologii badawczej przyjętej w pracy magisterskiej. Celem jest zapewnienie transparentności i powtarzalności przeprowadzonych eksperymentów, a także umożliwienie obiektywnej oceny uzyskanych wyników. Zaprezentowana metodologia stanowi kompleksowe podejście do weryfikacji postawionych hipotez badawczych oraz odpowiedzi na zdefiniowane problemy. Rozdział rozpoczyna się od precyzyjnego sformułowania problemu badawczego i szczegółowych pytań badawczych, które wyznaczają kierunek empirycznej części pracy. Następnie omówione zostaną kluczowe aspekty projektowania eksperymentów, w tym wybór odpowiednich metryk, scenariuszy obciążenia oraz specyfika środowiska testowego. Całość ma na celu stworzenie solidnych ram dla rzetelnej analizy wydajności i efektywności autonomicznych pętli sterowania w środowisku Kubernetes.

% % % % % % % % % % % % % % % % % % % % % % % % % % % % % % % % % % % % % % % % % % % % % % % % % % % % % % % %

\subsection{Sformułowanie problemu badawczego i pytań badawczych}

Głównym problemem badawczym, do którego rozwiązania dąży niniejsza praca, jest określenie optymalnych 
strategii autonomicznego zarządzania cyklem życia usług w środowisku Kubernetes, ze szczególnym uwzględnieniem 
mechanizmów skalowania i alokacji zasobów. W kontekście rosnącej złożoności aplikacji rozproszonych i potrzeby 
minimalizacji interwencji ludzkiej, kluczowe staje się zrozumienie, w jaki sposób istniejące rozwiązania radzą 
sobie z dynamicznie zmieniającymi się warunkami obciążenia oraz czy autorskie podejście, oparte na modelu 
MAPE-K, może zaoferować wyższą efektywność i adaptacyjność. Problem ten dotyka bezpośrednio wyzwań związanych 
z utrzymaniem wysokiej dostępności, optymalnego wykorzystania zasobów oraz obniżeniem kosztów operacyjnych w 
nowoczesnych architekturach chmurowych.

W celu rozwiązania postawionego problemu badawczego, sformułowano następujące szczegółowe pytania badawcze:
\begin{enumerate}

    \item \textbf{Charakterystyka i Ograniczenia Istniejących Mechanizmów}: W jaki sposób \gls{vpa}, \gls{hpa}
    oraz \gls{keda} reagują na różnorodne, dynamiczne wzorce obciążenia (np. narastające obciążenie, 
    obciążenie szczytowe, obciążenie zmienne cyklicznie) w kontekście efektywności alokacji zasobów 
    (CPU, pamięć) i stabilności usług (opóźnienia, błędy)? Jakie są ich fundamentalne ograniczenia w zakresie 
    proaktywnego skalowania i optymalizacji, które wynikają z ich wewnętrznych zasad działania?

    \item \textbf{Wydajność i Adaptacyjność Rozwiązania MAPE-K}: Czy autorskie rozwiązanie autonomicznej pętli 
    sterującej, zaprojektowane i zaimplementowane w oparciu o model \gls{mapek}, jest w stanie osiągnąć wyższą 
    efektywność wykorzystania zasobów oraz lepszą responsywność (np. niższe średnie i maksymalne opóźnienia) 
    aplikacji w środowisku Kubernetes w porównaniu do \gls{vpa}, \gls{hpa} i \gls{keda}, szczególnie w 
    scenariuszach wymagających zaawansowanej predykcji i koordynacji decyzji?

    \item \textbf{Wpływ Bazy Wiedzy na Efektywność MAPE-K}: W jakim stopniu komponent bazy wiedzy w 
    architekturze \gls{mapek}, integrujący historyczne dane o wydajności i predefiniowane polityki adaptacji, 
    przyczynia się do poprawy jakości decyzji podejmowanych przez pętlę sterującą oraz do długoterminowej 
    optymalizacji działania systemu w porównaniu do podejść czysto reaktywnych?

    \item \textbf{Kryteria Oceny i Metryki Sukcesu}: Jakie kluczowe metryki wydajnościowe (np. średnie zużycie
    CPU/RAM, procent niewykorzystanych zasobów, czasy odpowiedzi, przepustowość, liczba błędów) oraz wskaźniki
    operacyjne (np. liczba restartów Podów, stabilność skalowania) należy zastosować, aby obiektywnie porównać 
    skuteczność działania \gls{vpa}, {hpa}, \gls{keda} oraz autorskiego rozwiązania \gls{mapek} w 
    kontrolowanych warunkach eksperymentalnych?

\end{enumerate}


% % % % % % % % % % % % % % % % % % % % % % % % % % % % % % % % % % % % % % % % % % % % % % % % % % % % % % % %

\subsection{Środowisko testowe}

\subsubsection{Warstwa wirtualizacji i sprzętu}
W celu przeprowadzenia badań oraz weryfikacji przyjętych założeń projektowych przygotowano 
dedykowane środowisko testowe. Infrastruktura została zaprojektowana w oparciu o paradygmat 
\gls{iac}, co zapewnia powtarzalność procesu wdrażania oraz spójność konfiguracji.
Jako platformę bazową wykorzystano środowisko wirtualizacji Proxmox. Na jego potrzeby 
powołano klaster składający się z trzech maszyn wirtualnych o identycznej specyfikacji 
sprzętowej. Każdy z węzłów dysponuje następującymi zasobami:
\begin{itemize}
    \item Procesor (vCPU): 4 rdzenie, \item Pamięć operacyjna (RAM): 8 GB,
    \item Przestrzeń dyskowa: 100 GB.
\end{itemize}

Proces powoływania maszyn wirtualnych oraz konfiguracja warstwy sprzętowej zostały 
zautomatyzowane przy użyciu narzędzia Terraform.

\subsubsection{System operacyjny i orkiestracja}
Na wszystkich węzłach zainstalowano system operacyjny Ubuntu 24.04 LTS. Stanowi on bazę 
dla klastra Kubernetes, który został wdrożony przy użyciu lekkiej dystrybucji k3s w wersji 
\texttt{v1.33.2+k3s1}.

Instalacja oraz wstępna konfiguracja klastra zostały zrealizowane za pomocą narzędzia do 
zarządzania konfiguracją Ansible, z wykorzystaniem roli \texttt{k3s-ansible} dostępnej w 
repozytorium \footnote{\url{https://github.com/k3s-io/k3s-ansible}}. W ramach warstwy 
sieciowej klastra \gls{cni} zastosowano domyślne rozwiązanie dla k3s, czyli Flannel, co 
zapewnia stabilną komunikację między podami wewnątrz klastra.

Zarządzanie cyklem życia aplikacji oraz konfiguracją klastra odbywa się zgodnie z podejściem 
GitOps. Kluczowym elementem tej architektury jest narzędzie FluxCD, które stale monitoruje 
repozytorium kodu i automatycznie synchronizuje stan klastra z definicjami zawartymi w 
systemie kontroli wersji.

Proces dostarczania oprogramowania wspierany jest przez przygotowane potoki \gls{cicd},
które odpowiadają za budowanie, testowanie oraz przygotowanie manifestów wdrożeniowych. 
Takie podejście pozwoliło na pełną automatyzację procesu wdrażania kolejnych zasobów i 
usług w środowisku badawczym.

\begin{table}[h]
    \centering
    \caption{Specyfikacja techniczna środowiska eksperymentalnego}
    \label{tab:srodowisko_testowe}
    \vspace{0.3cm}
    \begin{tabular}{@{}ll@{}}
        %\toprule
        \textbf{Kategoria} & \textbf{Komponent / Specyfikacja} \\
        %\midrule
        Platforma wirtualizacji & Proxmox \\
        Automatyzacja infrastruktury & Terraform \\
        Konfiguracja systemu & Ansible (\texttt{k3s-ansible}) \\
        %\midrule
        Liczba węzłów & 3 maszyny wirtualne \\
        Zasoby węzła & 4 vCPU, 8 GB RAM, 100 GB HDD \\
        System operacyjny & Ubuntu 24.04 LTS \\
        %\midrule
        Orkiestrator & k3s (\texttt{v1.33.2+k3s1}) \\
        Sieć (CNI) & Flannel \\
        Model wdrażania & GitOps (FluxCD) \\
        %\midrule
        Narzędzie obciążające & k6 (uruchamiane zewnętrznie) \\
        Metodyka testów & Generowanie ruchu HTTP z konsoli lokalnej \\
        %\bottomrule
    \end{tabular}
\end{table}

\subsubsection{Instrumentarium badawcze i narzędzia pomiarowe}
W celu realizacji testów wydajnościowych oraz weryfikacji stabilności klastra pod obciążeniem,
zastosowano dedykowane narzędzia do generowania ruchu sieciowego.

Jako generator obciążenia wykorzystano narzędzie \textbf{k6}, które służy do przeprowadzania 
testów wydajnościowych. Skrypty testowe k6, przygotowane w języku JavaScript, są uruchamiane 
lokalnie z konsoli na stanowisku roboczym, a generowane przez nie żądania HTTP kierowane są 
na zewnątrz klastra Kubernetes, co realistycznie symuluje ruch pochodzący od zewnętrznych 
użytkowników. Taka metodyka pozwala na mierzenie kluczowych metryk z perspektywy klienta, w 
tym opóźnień (\textit{latency}), przepustowości (\textit{throughput}) oraz wskaźnika błędów.

Dane obciążeniowe kierowane są na docelową aplikację testową, która została zaprojektowana 
specjalnie na potrzeby pracy magisterskiej. Architektura i specyfikacja tej aplikacji zostały 
szczegółowo opisane w kolejnej sekcji dotyczącej eksperymentów.



% % % % % % % % % % % % % % % % % % % % % % % % % % % % % % % % % % % % % % % % % % % % % % % % % % % % % % % %

%%%%%%%%%%%%%%%%%%%%%%%%%%%%%%%%%%%%%%%%%%%%%%%%%%%%%%%%%%%%%%%%%%%%%%%%%%%%%%%%%%%%%%%%%%%%%%

\subsection{Projektowanie eksperymentów}
Projektowanie eksperymentów stanowi kluczowy etap niniejszej pracy magisterskiej, mający 
na celu systematyczną i powtarzalną weryfikację postawionych hipotez oraz udzielenie 
odpowiedzi na zdefiniowane problemy badawcze. Celem jest obiektywna ocena wydajności, 
efektywności oraz zachowania autonomicznych mechanizmów skalowania w środowisku Kubernetes. 
Eksperymenty zostaną zaprojektowane tak, aby umożliwić bezpośrednie porównanie działania 
istniejących rozwiązań (VPA, HPA, KEDA) z autorskim podejściem opartym na modelu MAPE-K, 
w kontrolowanych warunkach obciążenia.

%%%%%%%%%%%%%%%%%%%%%%%%%%%%%%%%%%%%%%%%%%%%%%%%%%%%%%%%%%%%%%%%%%%%%%%%%%%%%%%%%%%%%%%%%%%%%%

\subsubsection{Wybór metryk oceny}
Aby rzetelnie ocenić skuteczność poszczególnych mechanizmów autoskalowania, konieczny jest 
precyzyjny wybór obiektywnych i mierzalnych metryk. W ramach niniejszych badań zostaną 
uwzględnione następujące kategorie metryk, które pozwolą na wieloaspektową analizę:

\begin{enumerate}
    \item \textbf{Metryki efektywności wykorzystania zasobów:}
    \begin{itemize}
        \item \textbf{Średnie i maksymalne zużycie CPU (\%):} Procentowe wykorzystanie 
        procesora przez Pody w stosunku do przydzielonych im zasobów. Wysokie zużycie 
        przy niskim marginesie błędu świadczy o efektywnym wykorzystaniu.
        \item \textbf{Średnie i maksymalne zużycie pamięci (MiB/\%):} Analogicznie, pomiar 
        wykorzystania pamięci.
        \item \textbf{Współczynnik marnotrawstwa zasobów (\%):} Obliczony jako stosunek 
        niewykorzystanych, ale przydzielonych zasobów do całkowitej ilości przydzielonych 
        zasobów. Niższy współczynnik wskazuje na lepszą optymalizację.
        \item \textbf{Liczba alokowanych vs. faktycznie wykorzystanych zasobów:} Analiza 
        różnic między zasobami żądanymi (requests) i limitowanymi (limits) a rzeczywistym 
        zużyciem.
    \end{itemize}
    \item \textbf{Metryki wydajności usług (Quality of Service - QoS):}
    \begin{itemize}
        \item \textbf{Średni i maksymalny czas odpowiedzi (Response Time - RT):} Mierzone 
        w milisekundach, od momentu wysłania żądania do otrzymania odpowiedzi. Niższy czas 
        odpowiedzi wskazuje na lepszą responsywność.
        \item \textbf{Przepustowość (Throughput):} Liczba zrealizowanych operacji 
        (np. żądań HTTP) na jednostkę czasu. Wyższa przepustowość oznacza większą zdolność 
        systemu do obsługi obciążenia.
        \item \textbf{Wskaźnik błędów (\%):} Procent nieudanych żądań lub operacji w stosunku 
        do całkowitej liczby. Niższy wskaźnik świadczy o stabilności.
        \item \textbf{P95/P99 czasu odpowiedzi:} 95. i 99. percentyl czasu odpowiedzi, 
        wskazujący na doświadczenia większości użytkowników i skalę problemów dla marginalnej 
        grupy.
    \end{itemize}
    \item \textbf{Metryki operacyjne i stabilności:}
    \begin{itemize}
        \item \textbf{Liczba skalowań (up/down):} Częstotliwość zmian liczby replik lub 
        zasobów, co może świadczyć o stabilności lub "chwiejności" autoskalera.
        \item \textbf{Liczba restartów Podów:} Dotyczy VPA, gdzie zmiana zasobów może wymagać 
        restartu, co wpływa na dostępność.
        \item \textbf{Czas stabilizacji po zmianie obciążenia:} Czas potrzebny systemowi na 
        osiągnięcie stabilnego stanu (np. powrót metryk do pożądanych wartości) po gwałtownej 
        zmianie obciążenia.
    \end{itemize}
\end{enumerate}
Zebrane dane zostaną poddane analizie statystycznej w celu wyciągnięcia wiarygodnych wniosków.

%%%%%%%%%%%%%%%%%%%%%%%%%%%%%%%%%%%%%%%%%%%%%%%%%%%%%%%%%%%%%%%%%%%%%%%%%%%%%%%%%%%%%%%%%%%%%%

\subsubsection{Scenariusze obciążenia i symulacje}
Aby zapewnić kompleksową ocenę i porównanie mechanizmów skalowania, eksperymenty zostaną 
przeprowadzone z wykorzystaniem różnorodnych, realistycznych \textbf{scenariuszy obciążenia}. 
Scenariusze te mają za zadanie odzwierciedlać typowe wzorce ruchu występujące w rzeczywistych 
środowiskach produkcyjnych, a także testować odporność systemów na nagłe i ekstremalne zmiany. 
Planuje się zastosowanie następujących typów obciążenia:

\begin{itemize}
    \item \textbf{Obciążenie stałe (Constant Load):} Utrzymywanie stabilnego, umiarkowanego 
    obciążenia przez dłuższy czas, w celu oceny efektywności bazowej alokacji zasobów i 
    stabilności działania autoskalerów.
    \item \textbf{Obciążenie narastające (Ramp-up Load):} Stopniowe zwiększanie obciążenia, 
    aby zbadać zdolność autoskalerów do adaptacji i skalowania w górę w miarę wzrostu 
    zapotrzebowania.
    \item \textbf{Obciążenie szczytowe (Peak Load):} Gwałtowny i wysoki wzrost obciążenia 
    (tzw. "spike"), mający na celu przetestowanie szybkości reakcji i odporności mechanizmów 
    na nagłe, intensywne zapotrzebowanie.
    \item \textbf{Obciążenie zmienne cyklicznie (Cyclic/Diurnal Load):} Symulacja wzorców 
    ruchu charakterystycznych dla cykli dobowych lub tygodniowych (np. niższe obciążenie w 
    nocy, wyższe w ciągu dnia), w celu oceny zdolności autoskalerów do efektywnego skalowania 
    zarówno w górę, jak i w dół.
    \item \textbf{Obciążenie ze zmiennymi wzorcami dostępu (Varying Access Patterns):} 
    Złożone scenariusze, które mogą symulować różne typy operacji (np. odczyty vs. zapisy 
    w bazie danych), aby ocenić, jak mechanizmy radzą sobie z różnorodnym zapotrzebowaniem 
    na zasoby.
\end{itemize}
Każdy scenariusz zostanie powtórzony wielokrotnie w celu zapewnienia wiarygodności 
statystycznej wyników.

%%%%%%%%%%%%%%%%%%%%%%%%%%%%%%%%%%%%%%%%%%%%%%%%%%%%%%%%%%%%%%%%%%%%%%%%%%%%%%%%%%%%%%%%%%%%%%

\subsubsection{Narzędzia i środowisko eksperymentalne}
Przeprowadzenie rzetelnych eksperymentów wymaga odpowiednio skonfigurowanego środowiska testowego 
oraz zestawu narzędzi. W niniejszej pracy zostanie wykorzystany następujący zestaw zasobów:

\paragraph{Obciążana usługa testowa}
\hfill\\
Obciążana usługa testowa została zaimplementowana w języku Python z wykorzystaniem frameworka 
\textbf{FastAPI}. Jej głównym zadaniem jest symulowanie zróżnicowanych scenariuszy zużycia 
zasobów, niezbędnych do weryfikacji funkcjonalności mechanizmów automatycznego skalowania
(w tym VPA, HPA, KEDA oraz Operatora MAPE-K). Aplikacja udostępnia trzy dedykowane punkty 
końcowe (endpointy), z których każdy symuluje inny typ obciążenia:

\begin{itemize}
    \item \textbf{\texttt{/cpu}}: Symuluje obciążenie \textit{CPU-bound} (zależne od procesora) 
    poprzez wykonywanie operacji haszujących (\texttt{hashlib.sha256}).
    \item \textbf{\texttt{/mem}}: Symuluje obciążenie \textit{Memory-bound} (związane z 
    pamięcią operacyjną) poprzez dynamiczną alokację dużych tablic (\texttt{numpy}).
    \item \textbf{\texttt{/io}}: Symuluje obciążenie \textit{I/O-bound} (wejścia/wyjścia) 
    poprzez operacje zapisu i odczytu plików tymczasowych.
\end{itemize}

Poprzez zmianę parametrów przekazywanych w żądaniach HTTP do tych endpointów, możliwe jest 
precyzyjne kontrolowanie intensywności obciążenia klastra. Ponadto, aplikacja posiada 
wbudowaną instrumentację (\texttt{prometheus-fastapi-instrumentator}), która automatycznie 
wystawia metryki wydajnościowe. Metryki te są zbierane przez system monitorujący klastra, 
stanowiąc kluczowe dane wejściowe dla wszystkich testowanych metod automatycznego skalowania.

\paragraph{Generator ruchu k6}
\hfill\\
Jako generator ruchu sieciowego i narzędzie do przeprowadzania testów wydajnościowych 
wykorzystano \textbf{k6}. Jest to narzędzie \textit{open source}, które pozwala na 
definiowanie skomplikowanych scenariuszy testowych w języku JavaScript.

Kluczową cechą metodologiczną jest fakt, że \textbf{skrypty testowe k6 są uruchamiane lokalnie 
z konsoli} na stanowisku roboczym. Generowane przez nie żądania HTTP kierowane są na zewnętrzny 
adres klastra Kubernetes (przez serwis typu \texttt{LoadBalancer}), co realistycznie symuluje 
ruch pochodzący od zewnętrznych użytkowników. Taka konfiguracja umożliwia pomiar kluczowych 
metryk (np. opóźnienia \textit{latency}) z perspektywy klienta, zapewniając wiarygodność 
danych wyjściowych testów porównawczych.

\paragraph{Środowisko eksperymentalne (klaster Kubernetes)}
\hfill\\
Szczegółowa specyfikacja klastra Kubernetes, w tym parametry maszyn wirtualnych, system 
operacyjny oraz narzędzia do zarządzania infrastrukturą (\textbf{Terraform}, \textbf{Ansible}, 
\textbf{FluxCD}), zostały opisane w podrozdziałach \texttt{3.2.1} i \texttt{3.2.2}. Na potrzeby 
eksperymentów w tym środowisku wdrożono również kompletny stos monitorujący (\textbf{Prometheus}, 
\textbf{Grafana}), który służy zarówno do zasilania pętli Operatora MAPE-K, jak i jako główne 
narzędzie do analizy danych i wizualizacji wyników porównawczych wszystkich testowanych 
mechanizmów skalowania (VPA, HPA, KEDA).


%%%%%%%%%%%%%%%%%%%%%%%%%%%%%%%%%%%%%%%%%%%%%%%%%%%%%%%%%%%%%%%%%%%%%%%%%%%%%%%%%%%%%%%%%%%%%%

\subsubsection{Analiza Wrażliwości i Narzut Systemowy (Overhead)}

Weryfikacja efektywności autonomicznej pętli sterującej wymaga nie tylko pomiaru zysków w zakresie wydajności (\textit{Service Level Objectives - SLO}) i optymalizacji kosztów, lecz także krytycznej analizy narzutu systemowego (\textit{overhead}) generowanego przez sam mechanizm monitorowania i sterowania. Agresywne nastawy pętli, w szczególności interwał odpytywania metryk (\textit{polling rate}) ustawiony na $1\,\text{s}$ (w przypadku \texttt{KEDA} i \texttt{Prometheus}), stanowią kluczowy czynnik ryzyka, wprowadzający potencjalną niestabilność i przeciążenie.

Narzut systemowy, $\mathcal{O}_{\text{MAPE-K}}$, definiowany jest jako zużycie zasobów klastra, które jest bezpośrednim skutkiem działania pętli sterującej, a nie obciążenia aplikacyjnego. W ramach pracy wyodrębniono trzy główne wektory generowania narzutu przy wysokiej częstotliwości próbkowania:

\begin{enumerate}
    \item \textbf{Saturacja Warstwy Sterowania (\textit{Control Plane Saturation}):} Częste zapytania (\texttt{GET/LIST} zasobów \texttt{CustomResourceDefinitions} oraz statusów \texttt{HPA/Deployment}) do \texttt{kube-apiserver} i \texttt{etcd}. Jest to najbardziej krytyczny punkt obciążenia, prowadzący do dławienia (\textit{throttling}) pozostałych żądań administracyjnych.
    \item \textbf{Koszty Infrastruktury Monitorującej:} Wzrost zużycia CPU i pamięci RAM przez komponenty takie jak \texttt{Prometheus} (ze względu na konieczność parsowania, indeksowania i zapisu dużej liczby próbek na sekundę) oraz operatorzy skalujący (\texttt{KEDA Operator}).
    \item \textbf{Narzut Aplikacyjny i Sieciowy:} Konieczność częstego generowania i serializacji metryk wewnątrz aplikacji (endpoint \texttt{/metrics}), co w systemach jednowątkowych może prowadzić do wzrostu opóźnień (\textit{latency}) w obsłudze ruchu biznesowego.
\end{enumerate}

Do pomiaru narzutu systemowego wykorzystano zestaw metryk bazujących na systemach monitorujących klastra Kubernetes:
\begin{itemize}
    \item \textbf{Zużycie CPU/RAM \texttt{kube-apiserver}:} Monitorowane za pomocą metryk \texttt{kube\_pod\_container\_resource\_limits} oraz \texttt{usage\_seconds\_total}.
    \item \textbf{Współczynnik Dławienia (\textit{Throttling Rate}):} Wzrost liczby błędów \texttt{429 (Too Many Requests)} w metryce \texttt{apiserver\_request\_total}.
    \item \textbf{Opóźnienie etcd:} Analiza metryki \texttt{etcd\_disk\_wal\_fsync\_duration\_seconds} jako wskaźnika wydajności persystencji danych klastra.
\end{itemize}


W celu wyznaczenia opłacalności stosowania agresywnego próbkowania przeprowadzono test A/B, porównując dwie konfiguracje pętli sterującej przy identycznym obciążeniu syntetycznym:

\begin{itemize}
    \item \textbf{Scenariusz A (Baseline):} \texttt{Polling Rate} $= 30\,\text{s}$ (Standardowy).
    \item \textbf{Scenariusz B (Real-time):} \texttt{Polling Rate} $= 1\,\text{s}$ (Agresywny).
\end{itemize}
Wyniki testu umożliwią określenie punktu, w którym narzut $\mathcal{O}_{\text{MAPE-K}}$ staje się istotny, a dalsze zwiększanie częstotliwości próbkowania przestaje przynosić korzyści w stabilizacji wydajności aplikacji.


% % % % % % % % % % % % % % % % % % % % % % % % % % % % % % % % % % % % % % % % % % % % % % % % % % % % % % % %

\subsection{Metody analizy danych}

Po zakończeniu fazy gromadzenia danych z przeprowadzonych eksperymentów, kluczowe jest zastosowanie odpowiednich metod analizy, które umożliwią wyciągnięcie wiarygodnych wniosków i weryfikację postawionych hipotez badawczych. Proces analizy danych będzie obejmował zarówno podejścia ilościowe, jak i jakościowe, a także wykorzystanie specjalistycznych narzędzi do wizualizacji i interpretacji wyników. Celem jest nie tylko stwierdzenie różnic w wydajności poszczególnych mechanizmów skalowania, ale również zrozumienie przyczyn obserwowanych zachowań.

\subsubsection{Analiza ilościowa i jakościowa}
Analiza danych zostanie przeprowadzona dwutorowo, łącząc metody ilościowe i jakościowe w celu uzyskania kompleksowego obrazu funkcjonowania badanych mechanizmów:


\begin{enumerate}
    \item \textbf{Analiza Ilościowa:} Będzie stanowiła podstawę weryfikacji hipotez i opierać się na statystycznym przetwarzaniu zebranych metryk numerycznych. Kluczowe aspekty analizy ilościowej obejmują:
    \begin{itemize}
        \item \textbf{Statystyka opisowa:} Obliczenie średnich, median, odchyleń standardowych, minimum i maksimum dla wszystkich zebranych metryk (np. czasu odpowiedzi, zużycia CPU, przepustowości). Pozwoli to na wstępne scharakteryzowanie danych i identyfikację rozkładów.
        \item \textbf{Analiza trendów:} Badanie zmian metryk w czasie dla różnych scenariuszy obciążenia, co pozwoli na ocenę dynamiki reakcji każdego autoskalera.
        \item \textbf{Analiza korelacji:} Zbadanie związków między różnymi metrykami (np. między obciążeniem a czasem odpowiedzi, lub między liczbą Podów a zużyciem zasobów).
        \item \textbf{Testy porównawcze:} Zastosowanie odpowiednich testów statystycznych (np. testów t-Studenta, analizy wariancji ANOVA) do porównania średnich wartości metryk między poszczególnymi mechanizmami skalowania (VPA, HPA, KEDA, MAPE-K) w celu określenia istotności statystycznej obserwowanych różnic.
        \item \textbf{Analiza percentylowa:} Obliczenie percentyli (np. P95, P99) dla metryk wydajności, co jest kluczowe dla oceny doświadczeń użytkowników i identyfikacji ewentualnych anomalii lub "ogonów" rozkładu.
    \end{itemize}
    \item \textbf{Analiza Jakościowa:} Uzupełni analizę ilościową, dostarczając głębszego zrozumienia przyczyn obserwowanych zachowań, zwłaszcza w przypadku anomalii lub niespodziewanych wyników. Obejmie ona:
    \begin{itemize}
        \item \textbf{Analiza logów systemowych i zdarzeń Kubernetesa:} Przeglądanie logów z komponentów autoskalerów, kubeleta i kontrolera Kubernetes w celu zidentyfikowania konkretnych decyzji, błędów lub zdarzeń, które mogły wpłynąć na zachowanie systemu.
        \item \textbf{Analiza konfiguracji i polityk:} Szczegółowa ocena, jak specyficzne konfiguracje i algorytmy każdego autoskalera wpływały na jego reakcje w danych scenariuszach.
        \item \textbf{Studia przypadków (Case Studies):} Wybór konkretnych, interesujących momentów w przebiegu eksperymentów (np. nagłe skoki obciążenia, awarie, momenty stabilizacji) i ich dogłębna analiza jakościowa w celu wyjaśnienia obserwowanych zjawisk.
    \end{itemize}
\end{enumerate}
Połączenie obu podejść pozwoli na kompleksową interpretację wyników, łącząc precyzję danych numerycznych z kontekstualnym zrozumieniem mechanizmów działania.

\subsubsection{Narzędzia do wizualizacji i interpretacji wyników}
Efektywna wizualizacja danych jest niezbędna do szybkiego zrozumienia złożonych zbiorów danych i komunikowania wyników badań. Do wizualizacji i interpretacji wyników eksperymentów zostaną wykorzystane następujące narzędzia:

\begin{itemize}
    \item \textbf{Grafana:} To narzędzie typu open-source do tworzenia interaktywnych dashboardów. Zostanie użyta do wizualizacji metryk czasowych (np. zużycie CPU/RAM w funkcji czasu, liczba replik, czasy odpowiedzi) zbieranych z Prometheus. Pozwoli to na dynamiczne śledzenie zmian i porównywanie zachowań różnych autoskalerów.
    \item \textbf{Jupyter Notebooks (z bibliotekami Python, np. Pandas, Matplotlib, Seaborn):} Środowisko to zostanie wykorzystane do zaawansowanej analizy statystycznej i generowania niestandardowych wykresów. Umożliwi ono szczegółowe przetwarzanie danych, przeprowadzanie testów statystycznych oraz tworzenie wysokiej jakości wizualizacji (np. wykresów słupkowych, liniowych, pudełkowych, rozrzutu), które zostaną włączone do pracy dyplomowej.
    \item \textbf{Arkusz kalkulacyjny (np. Microsoft Excel, Google Sheets):} Do wstępnej analizy danych, ich agregacji oraz prostych obliczeń statystycznych. Może być również wykorzystany do organizacji surowych danych.
    \item \textbf{Narzędzia do analizy logów (np. Loki, ELK Stack):} Służą do przeszukiwania, agregacji i analizy logów systemowych, co jest kluczowe dla jakościowej oceny zachowania autoskalerów i wykrywania problemów.
\end{itemize}
Wykorzystanie tych narzędzi zapewni zarówno możliwość dogłębnej analizy statystycznej, jak i czytelną prezentację uzyskanych rezultatów, co jest kluczowe dla skutecznej komunikacji wniosków badawczych.



