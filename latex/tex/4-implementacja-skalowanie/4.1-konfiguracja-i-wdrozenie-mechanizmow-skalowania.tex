%%%%%%%%%%%%%%%%%%%%%%%%%%%%%%%%%%%%%%%%%%%%%%%%%%%%%%%%%%%%%%%%%%%%%%%%%%%%%%%%%%%%%%%%%%%%%%%%%%

\subsection{Konfiguracja i Wdrożenie Mechanizmów Skalowania}

W niniejszym podrozdziale przedstawiono proces konfiguracji referencyjnych mechanizmów skalowania 
(VPA, HPA, KEDA) w przygotowanym środowisku badawczym opartym na dystrybucji k3s. Zgodnie z 
przyjętą w pracy metodyką \textit{Infrastructure as Code}, wszystkie polityki skalowania zostały 
zdefiniowane w formie deklaratywnych manifestów YAML i wdrożone przy użyciu narzędzia FluxCD, co 
zapewnia powtarzalność eksperymentów oraz spójność konfiguracji z repozytorium git.

Celem tego etapu było nie tylko uruchomienie poszczególnych autoskalerów, ale przede wszystkim ich 
kalibracja w taki sposób, aby stanowiły wiarygodny punkt odniesienia (ang. \textit{baseline}) dla 
autorskiego rozwiązania opartego na modelu MAPE-K. W procesie konfiguracji szczególną uwagę 
zwrócono na parametry określające czułość mechanizmów oraz częstotliwość próbkowania metryk, 
dążąc do wyeliminowania zjawiska oscylacji (ang. \textit{flapping}) przy jednoczesnym zachowaniu 
responsywności na zadane scenariusze obciążenia (CPU-bound oraz Memory-bound). Poniżej opisano 
szczegóły implementacyjne dla każdego z badanych mechanizmów.

%%%%%%%%%%%%%%%%%%%%%%%%%%%%%%%%%%%%%%%%%%%%%%%%%%%%%%%%%%%%%%%%%%%%%%%%%%%%%%%%%%%%%%%%%%%%%%%%%%

\subsubsection{Konfiguracja i Wdrożenie VPA (Vertical Pod Autoscaler)}

Vertical Pod Autoscaler (VPA) został wdrożony w środowisku eksperymentalnym w celu automatycznego 
dostosowywania wartości \textit{requests} oraz \textit{limits} zasobów CPU i pamięci dla aplikacji 
\texttt{workload-app}. Mechanizm ten, zgodnie z architekturą opisaną w rozdziale poświęconym 
podstawom teoretycznym, składa się z trzech komponentów: \textit{Recommender}, \textit{Updater} 
oraz \textit{Admission Controller}, które odpowiednio analizują historyczne metryki, stosują 
rekomendacje i nadpisują konfigurację zasobów podczas tworzenia Podów.

Wdrożenie VPA przeprowadzono zgodnie z metodyką \textit{Infrastructure as Code}, w pełni 
zintegrowaną z podejściem GitOps wykorzystującym narzędzie FluxCD. Wszystkie manifesty zostały 
zapisane w repozytorium i automatycznie synchronizowane ze stanem klastra. Taki sposób zarządzania 
pozwolił na zapewnienie powtarzalności eksperymentów oraz ścisłej kontroli nad wersjonowaniem 
konfiguracji.

Konfiguracja VPA dla aplikacji \texttt{workload-app} obejmowała utworzenie dedykowanego obiektu 
\texttt{VerticalPodAutoscaler}, który obserwuje zasoby zużywane przez kontener oraz generuje 
rekomendacje dotyczące ich optymalnego przydziału. W ramach eksperymentów zastosowano tryb 
działania \textit{``Auto''}, co oznacza, że komponent \textit{Updater} mógł w sposób automatyczny 
restartować Pody, aby wprowadzić nowe wartości zasobów, gdy zostały one uznane za bardziej 
adekwatne na podstawie dotychczasowych pomiarów.

Fragment przykładowej konfiguracji VPA przedstawiono poniżej:

VPA został uruchomiony równolegle z pozostałymi mechanizmami skalowania, jednak nie wpływał 
bezpośrednio na liczbę replik aplikacji. Jego zadaniem było wyłącznie dostosowywanie zasobów 
przypisanych pojedynczemu Podowi, co umożliwiało analizę wpływu zmian wertykalnych na wskaźniki 
wydajności oraz stabilności systemu. Zastosowanie tego mechanizmu było szczególnie istotne w 
scenariuszach \textit{memory-bound}, gdzie właściwe dobranie limitów miało kluczowe znaczenie dla 
uniknięcia zjawiska \textit{Out-of-Memory Killing} oraz nadmiernej alokacji pamięci.

Wdrożenie VPA w środowisku k3s, wspieranym przez Flannel jako warstwę sieciową, nie wymagało 
dodatkowej konfiguracji w obrębie klastra. Komponenty autoskalera zostały zainstalowane jako 
osobny zestaw CRD i kontrolerów, a ich działanie monitorowane było za pomocą Prometheus i Grafana, 
zgodnie z narzędziami pomiarowymi opisanymi w rozdziale metodologicznym. Dzięki temu cały proces 
obserwacji i oceny działania VPA mógł zostać przeprowadzony w sposób systematyczny oraz spójny 
z pozostałymi eksperymentami.


\subsubsection{Konfiguracja i Wdrożenie HPA (Horizontal Pod Autoscaler)}

\subsubsection{Konfiguracja i Wdrożenie KEDA (Kubernetes Event-driven Autoscaling)}
